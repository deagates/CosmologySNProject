\documentclass[aps,prl,reprint]{revtex4-1}
\usepackage{blindtext}
\usepackage{ mathrsfs }
\usepackage{amsmath}
%\usepackage[justification=centering]{caption}
\usepackage{graphics,setspace,enumitem,graphicx,textpos}
\begin{document}
\bibliographystyle{plain}
\title{	Analysis of Matter and Dark Energy Content of the Universe via MCMC fitting of Type Ia Supernova Data}
\author{Delilah Gates}
\email{dgates@g.harvard.edu}
\author{Ann Wang}
\email{annwang@g.harvard.edu}
\affiliation{Harvard University}
%\begin{spacing}{1.2}
\begin{abstract}
Something about cosmology abstract TBA
\end{abstract}
\maketitle
\section{I. Introduction}
One compelling question in cosmology research today is why the energy budget of the universe is dominated by dark energy, or a cosmological constant. Since the observation of Type Ia supernovae showed that the universe was expanding at an accelerated rate, dark energy research has been an active area of study \cite{riess_sn}. Recently, a study by J.T. Nielson et al. \cite{shocker} claimed, analyzing a larger set of supernovae data, that there was little evidence for an accelerated rate of expansion. We reanalyze the same data set and present our findings here.
\section{II. Background}
To compare results, we use data from the Joint Lightcurve Analysis (JLA) catalogue \cite{sdss}, which uses the Spectral Adaptive Lightcurve Template 2 (SALT2) approach to characterize Type Ia Supernovae \cite{salt2}. There are several versions of the covariance matrix for the distance modulus; we use the covmat\_v6 version along with the corresponding code to convert the covariance data from FITS format \cite{fits} to a matrix format given $\alpha$, $\beta$ (see later $\mu$ equation), which was provided at http://supernovae.in2p3.fr/sdss\_snls\_jla/ReadMe.html. 
\par The data provides us with several characteristic values for every SN Ia: the heliocentric redshift $z_{hel}$, the CMB frame redshift $z_{CMB}$, the apparent magnitude $m_B$, the shape parameter from SALT2, $x_1$, the color correction $c$, and the host stellar mass $M_{stellar}$ \cite{sdss}. Using these parameters, we can define a distant modulus which can be compared to a value expected by a certain cosmological model. This distance modulus is given by \cite{sdss}: 
\begin{equation}
\mu = m_B - M + \alpha x_1 - \beta c
\end{equation}
The M, or absolute magnitude, is dependent on the host galaxy, so we follow the prescription in \cite{sdss}, where if $M_{stellar} < 10^{10} M_{\odot}$, then $M_B = M_B'$, otherwise, $M_B = M_B' + \Delta_M$. 

For comparison, using the density parameters ($\Omega_m$, $\Omega_{r}$, $\Omega_{\Lambda}$) and curvature ($\Omega_k$) of a comological model one can calculate the distance modulous of an object at a certain redshift as follows:  
\begin{align}
E(z)=&\sqrt{\Omega_{r}(1+z)^4 + \Omega_m(1+z)^3 + \Omega_k(1+z)^2 + \Omega_{\Lambda}} \nonumber \\
d_L(z)=&\frac{c}{H_0} {\int_0}^z \frac{dz'}{E(z')} \nonumber \\
\mu =& 25 + 5*log_{10} \left( \frac{d_L(z)}{Mpc} \right)
\end{align}
 

\section{III. Analysis}
We perform a likelihood analysis using MCMC methods. Our code can be found online at https://github.com/deagates/CosmologySNProject.
\par We will use two different likelihood functions: the first one, which we will call approach (a), is $$\mathscr{L} = exp(-\frac{\chi^2}{2}) $$ where$\chi^2 = (\hat{\mu}-\mu_{model})^\dagger C^{-1} (\hat{\mu}-\mu_{model})$ \cite{sdss}, where C is the covariance matrix, and $\hat{\mu}$ and $\mu_{model}$ are calculated using (1) and (2) respectively. The other approach (b), which is employed by J.T. Nielson et al., uses \begin{align*}\mathscr{L} = |2\pi(C+A^T \Sigma_l A)|^{-1/2}\; \\
\times \text{Exp}[-(\hat{Z}-Y_0A)(C+A^T\Sigma_lA)^{-1}(\hat{Z}-Y_0A)^T/2],\end{align*} where $\Sigma_l = diag[\sigma_{M0}^2,\sigma_{X_{1,0}}^2,\sigma_{C_0}^2,...],$ $A = ,$ $\hat{Z} = [\hat{m}_{B1}-\mu_1, \hat{x}_{11},\hat{c}_1,..]$, and $Y_0 = [M_0,x_{1,0},c_0,....]$. 

 For simplicity and comparability with J.T. Nielson et al we take $\Omega_r$ the radiation density parameter of the universe to be 0, further justified by the fact that the measured radiation density parameter of today is very small, and we assume the universe is flat ($\Omega_k = 0$). This leaves us with the constraint that $\Omega_m + \Omega_{\Lambda} = 1 $. This leaves us with five model parameters ($\Omega_m$, $\alpha$, $\beta$, $M'_B$, and $\Delta_M$) that our MCMC varies to try to minimize ${\chi}^2$ and thus maximize the likelihood.
 
\section{IV. Analysis} 
 
\bibliographystyle{apsrev4-1} % Tell bibtex which bibliography style to use
\bibliography{cosmo_bib} 
\end{document}
